\documentclass[10pt, oneside]{article} 
\usepackage{amsmath, amsthm, amssymb, calrsfs, wasysym, verbatim, bbm, color, graphics, geometry, cite}
\geometry{tmargin=.75in, bmargin=.75in, lmargin=.75in, rmargin = .75in}  

\newcommand{\R}{\mathbb{R}}
\newcommand{\C}{\mathbb{C}}
\newcommand{\Z}{\mathbb{Z}}
\newcommand{\N}{\mathbb{N}}
\newcommand{\Q}{\mathbb{Q}}
\newcommand{\Cdot}{\boldsymbol{\cdot}}
\newcommand{\I}{\mathbb{I}}
\newcommand{\HR}{\mathbb{HR}}
\newcommand{\HRI}{\mathbb{HRI}}
\newcommand{\sinp}{\mathrm{sinp}}
\newcommand{\cosp}{\mathrm{cosp}}
\newcommand{\cosi}{\mathrm{cosi}}
\newcommand{\sini}{\mathrm{sini}}
\newtheorem{thm}{Theorem}
\newtheorem{defn}{Definition}
\newtheorem{conv}{Convention}
\newtheorem{rem}{Remark}
\newtheorem{lem}{Lemma}
\newtheorem{cor}{Corollary}
\newtheorem{example}{Example}
\newtheorem{exe}{Exercise}
\newtheorem{conjecture}{Conjecture}
\title{Schrodinger Trig Functions}
\author{[Drew Remmenga]}

\begin{document}

\maketitle
\begin{abstract}
    This paper attempts to fill a hole in our understanding of trigonometric functions. Inspired by parabolic trigonometric functions and the schrodinger equation we define a new class of trigonometric functions. 
\end{abstract}
\section{As Differential Equations}
    In the context of differential equations we can define cosine as the unique solutions to the following system of differential equations. \cite{bartle1999introduction}
    \begin{align}
        \frac{d^2}{dx^2} f(x) = f(x), f(0) = 1, f'(0) = 0 \label{eq:1}
    \end{align}
    And we can define sine as the first derivative of cosine. 
    Similarly we may define the hyperbolic cosine function with this differential system. \cite{bartle1999introduction}
    \begin{align}
        \frac{d}{dx} f(x) = -f(x), f(0) = 1, f'(0) = 0 \label{eq:2}
    \end{align}
    Recently a new branch of trignometric functions were defined. These parabolic trigonometric functions parameterize the unit palabora. \cite{dattoli2011parabolictrigonometricfunctions} We can define the parabolic cosine with the following differential equation. 
    \begin{align}
        \frac{d^2}{dx^2} f(x) = 0, f(0) = 1, f'(0) = 0 \label{eq:3}
    \end{align}
    Now we can define a new class of functions using the following system.
    \begin{align}
        \frac{d^2}{dx^2} f(x) = i f(x), f(0) = 1, f'(0) = 0 \label{eq:4}
    \end{align}
\bibliographystyle{plain}  % or another style like alpha, unsrt, etc.
\bibliography{references.bib}  % the name of the .bib file
\end{document}

